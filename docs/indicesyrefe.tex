
%%%%%%%%%%%%%%%%%%%%%%%%%%%%%%%%%%%%%%%%%%%%%%%%%%%%%%%%%%%%%%%%%%%%%%%%%%%%
%%%%%%%%%%%%%%% ESTO ES PARA LA TABLA DE CONTENIDOS (INDICE)%%%%%%%%%%%%%%%%


\setcounter{secnumdepth}{3} %niveles en capitulos hasta (1.1.1)
\setcounter{tocdepth}{3} %niveles en indice hasta (1.1)


\addtocontents{toc}{~\hfill\small{Página}\par} %añade "pagina" a tabla de contanidos
\addtocontents{toc}{\vspace{2pt} \hrule \vspace{5mm} \par}

%\renewcommand\contentsname{Índice de contenidos}
\tableofcontents
\listoffigures % indice de figuras
\listoftables % indice de tablas
\lstlistoflistings % indice de listados



%---------------------------------------------------------------------------
% comienzo de relacion abreviaturas y/o acrónimos
%---------------------------------------------------------------------------


%\addcontentsline{toc}{section}{ABREVIATURAS}
\clearpage
\vspace{0.2cm}
\section*{ABREVIATURAS}

\begin{tabular}{ l   |    l  }
	

&\\
bpp & Bits per pixel \\ 
bps & Bits per second \\ 
CBG & Color Game Boy \\ 
DMG & Display Monochrome Game Boy \\ 
FIFO & First In First Out \\ 
FPU & Floating Point Unit \\ 
GBA & Game Boy Advance \\
OAM & Object Attribute Memory \\ 
ROM & Read-Only Memory \\ 
SIO & Serial Input/Output \\ 
VRAM & Video RAM \\
      &\\

\end{tabular}

\section*{GLOSARIO}

\begin{tabular}{ l   |    p{8cm}  }
	

&\\
HBlank & Simboliza el período que se produce entre cada línea refrescada por pantalla \\
Homebrew & Juegos o aplicaciones creadas por aficionados para plataformas cerradas \\
Renderizar & Proceso que involucra generar una imagen a partir de modelos (2D o 3D) y/o imágenes base \\ 
Sprint & El tiempo transcurrido para completar una nueva versión del producto \\
Sprites & Describe cualquier entidad única que se pueda mover dentro de un juego. Suelen ser personajes y objetos. \\
Tile & Pequeña imagen (en el caso de la Game Boy Advance, de 8x8) que sirve para posteriormente referenciarla en un mapa \\
Transformaciones de corte & Operación geométrica donde una parte del cuerpo se desliza paralelamente con respecto a otra \\
VBlank & Simboliza el período que se produce entre cada fotograma refrescado \\
VCount & Simboliza el número de líneas verticales refrescadas en la Game Boy Advance \\
      &\\

\end{tabular}

%\addcontentsline{toc}{section}{ABREVIATURAS}

